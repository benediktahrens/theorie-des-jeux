\documentclass[french,11pt,leqno]{article}

\usepackage{amsmath,amssymb,babel,a4wide,theorem}
\usepackage{enumerate}
%\usepackage[latin1]{inputenc}
\usepackage[charter]{mathdesign} 


%\pagestyle{empty}

\textheight 24 true cm
\textwidth 16.5 true cm
\topmargin -1cm
\oddsidemargin -3 truemm
\evensidemargin 3 truemm

\newcommand{\N}{\mathbb{N}}
\newcommand{\Z}{\mathbb{Z}}
\newcommand{\Q}{\mathbb{Q}}
\newcommand{\R}{\mathbb{R}}
\newcommand{\C}{\mathbb{C}}
\newcommand{\K}{\mathbb{K}}
\newcommand{\Rn}{\mathbb{R}^n}

\newcommand{\dsp}{\displaystyle}

\newcounter{exocount}
\setcounter{exocount}{1}
\newcounter{questcount}

\def\exo{\bigskip\noindent{\bf Exercice \theexocount {} -}
  \addtocounter{exocount}{1} \setcounter{questcount}{1}} 

\def\q{\smallskip\noindent{\bf \number\thequestcount -}
  \addtocounter{questcount}{1}}

\def\indic{\smallskip\noindent\textit{Indication.} }

\begin{document}

\begin{center}
{\bf Universit\'e de Nice-Sophia Antipolis}
\end{center}
\bigskip

\flushright

\noindent{\bf ANN\'EE UNIVERSITAIRE~:} 2011-12

\noindent{\bf ANN\'EE D'\'ETUDE~:} L1 MASS 

\noindent{\bf MATI\`ERE~:} THEORIE DES JEUX 

\flushleft

\vskip1cm

\noindent {\bf NOM:} \hspace{5cm} {\bf PR\'ENOM:} \hspace{4cm} {\bf NOMBRE FEUILLES:}


\vskip1cm

\begin{center}
{\bf  INTERROGATION DE TD 1}

{\bf 24 f\'evrier 2012 }
\end{center}

\medskip

{\bf \noindent Tous les documents et les calculatrices sont interdits.} 

\smallskip

{\bf \noindent Il vous est toujours demand\'e de justifier avec soin vos r\'eponses.} 

\medskip



\exo
On consid\`ere les formes normales suivantes avec $p\in \R$.
\begin{equation*}
\begin{array}{l|rrrr}
\mathbf{A} & y_1 & y_2 & y_3 & y_4 \\ \hline
x_1& 3&-1& 4 & 2\\
x_2&-1& 0 &3&4\\
x_3&2&-15&0&3\\
x_4&-1&2&3&7
\end{array}
% 
\qquad
%
\begin{array}{l|rrrr}
\mathbf{B}   & y_1 & y_2 & y_3 & y_4\\ \hline
x_1 &  3 & 9 & -13 & -3 \\
x_2 &  6 & 10 & -6 & -6 \\
x_3 &  6 & -3 & 6 & -3 \\
x_4 &  3 & -3 & -3 & 3 \\ 
\end{array}
%
\qquad
%
\begin{array}{l|rrr}
\mathbf{C}& y_1&y_2&y_3 \\ \hline
x_1& 1&1&p\\
x_2& -1&1&\sqrt{10}\\
x_3&\sqrt{8}&1& 1
\end{array}
\end{equation*}

\begin{enumerate}[a)]%\setlength{\itemsep}{-1ex}
\item D\'eterminer les strat\'egies domin\'ees des deux joueurs $X$ et $Y$ de $A$. Donner le jeu r\'eduit de $A$, qu'on appellera $A'$. 
      % A' il restent  x_1, x_3, y_1, y_2
\item Le jeu $A'$ poss\`ede--t--il des points selles ? 
           % y en a pas
\item Donner les strat\'egies prudentes de $A'$ pour chacun des joueurs.
\item Comment pourrait--on modifier la fonction de paiement de $A'$ pour obtenir un jeu \`a valeur en strat\'egie pure ? On modifiera
        une seule valeur de la fonction de paiement.
\item D\'eterminer les points selles de $B$. En d\'eduire les strat\'egies optimales de chacun des joueurs.
\item Selon la valeur de $p$, d\'eterminer les strat\'egies dominantes des deux joueurs $X$ et $Y$ de $C$.
        % b \leq 1  alors il restent que x_3, y_1    
        % b \geq 1   pas fait
\item D\'eterminer, pour quelle valeur de $p$, le jeu $C$ est \`a valeur en strat\'egie pure. Vous donnerez alors les points selles selon les valeurs de $p$.
        %   minmax = max(b,1)   =?=    min(b,1) =  maxmin 
        %   pour b = 1 !!!
\end{enumerate}



\exo 
On consid\`ere la forme normale $u_t$ suivante, pour $t$ un r\'eel :
\begin{equation*}
% \begin{pmatrix}
% \begin{array}{l|rr}
%     & y_1 & y_2  \\ \hline
% x_1 & t+1&1\\
% x_2 & 0&2t\\
% x_3 & -2&-1
% \end{array}
%
\begin{array}{l|rrr}
    & y_1 & y_2 & y_3  \\ \hline
x_1 & t-1&0 & 1\\
x_2 & 1&3t & 2\\

\end{array}
%
%
% \end{pmatrix}
\end{equation*}

\begin{enumerate}[a)]%\setlength{\itemsep}{-2ex}
\item Dessiner le graphe de la fonction $t \mapsto \inf_{y \in Y} \sup_{x \in X} u_t(x,y)$. On expliquera les diff\'erentes \'etapes effectu\'ees. \label{question:1}
\item Dessiner sur un autre graphe, la fonction $t \mapsto \sup_{x \in X} \inf_{y \in Y} u_t(x,y)$. \label{question:2}
\item D\'eduire des graphes des questions pr\'ec\'edentes les strat\'egies prudentes de X et Y en fonction de t. 
     On expliquera comment elles sont d\'etermin\'ees.
\item Dessiner sur un troisi\`eme graphe, les fonctions \[t \mapsto \inf_{y \in Y} \sup_{x \in X} u_t(x,y)\] et \[t \mapsto \sup_{x \in X} \inf_{y \in Y} u_t(x,y)\]
 obtenues lors des questions \ref{question:1} et \ref{question:2}. 
\item Pour quelles valeurs de $t$, le jeu a-t-il une valeur en strat\'egie pure et quelle est-elle ? On expliquera comment ces valeurs sont d\'etermin\'ees.
\item Selon les valeurs de $t$, d\'eterminer les strat\'egies dominantes pour $X$. On donnera les arguments utilis\'es.
% \item Selon les valeurs de $t$, d\'eterminer les strat\'egies dominantes pour $Y$. On donnera les arguments utilis\'es.
% TODO


\end{enumerate}



\exo
H\'el\`ene et Jimi jouent ensemble. Ils disposent chacun de trois cartes de couleurs diff\'erentes : rouge, bleu, jaune. 
Ils choisissent simultan\'ement une carte et la montrent. 
La carte rouge gagne sur la bleue qui gagne sur la jaune qui elle-m\^eme remporte sur la rouge. 
Si une carte rouge l'emporte, le joueur perdant donne 2 euros au gagnant. 
Pour les deux autres couleurs, le perdant donne 3 euros au gagnant. Si deux m\^emes couleurs sortent, on a \'egalit\'e.
\begin{enumerate}[a)]%\setlength{\itemsep}{0ex}
\item Donner la forme normale de ce jeu. Les diff\'erentes strat\'egies seront clairement explicit\'ees.
\item Ce jeu admet-il une valeur en strat\'egie pure ? Si oui, laquelle?
\item Ce jeu est--il \`a somme nulle ? Expliquer. Proposer une alternative pour avoir un jeu \`a somme non nulle.
\item Une autre version de ce jeu est propos\'ee. H\'el\`ene et Jimi doivent choisir deux cartes et les mettre dans deux pots distincts. 
     Dans le premier, ils doivent mettre celui dont la couleur est la premi\`ere dans l'ordre alphab\'etique et dans le second, l'autre carte. 
   Par exemple, si H\'el\`ene choisit la bleue et la jaune, bleu est le premier mot dans l'ordre alphab\'etique donc elle est plac\'e dans le premier pot et la 
  carte jaune dans le second pot. 
   Les gains de chacun des pots sont les m\^emes que pr\'ec\'edemment mais sont additionn\'es pour conna\^itre le gain de chacun \`a chaque partie. 
  Donner la forme de ce jeu en explicitant les diff\'erentes strat\'egies. Ce jeu admet--il une valeur en strat\'egie pure? Si oui, laquelle?
\end{enumerate}


\exo
Soit $J=(X,Y,u)$ un jeu \`a somme nulle. 
 \begin{enumerate}[a)]
  \item Quand le jeu $J$ admet--il une valeur en strat\'egie pure ? (Donner la d\'efinition du cours.)
  \item 
Donner la d\'efinition d'un point selle de $J$ donn\'ee en cours.
  \item Donner deux propri\'et\'es des points selles de $J$ discut\'ees en cours.
\end{enumerate}
\medskip
\medskip




\end{document}