\documentclass[12pt]{article}
%\documentclass[10pt,a4paper]{book}
\usepackage[utf8 ]{inputenc}
\usepackage[francais]{babel}
\usepackage{hyperref}
\usepackage{epsf}
\usepackage{amsmath}
\usepackage{epsfig}
\usepackage{amssymb}
\usepackage{epic}
\usepackage{stmaryrd}
\usepackage{graphicx}
\pagenumbering{arabic}
\usepackage{hyperref}
\usepackage{verbatim}
\usepackage{bm}
\usepackage[T1]{fontenc}
\usepackage{geometry}

\title{Th\'eorie des jeux : TD 1}
\date{}

\begin{document}
\pagestyle{empty}

\textbf{L1 MASS ;   Th\'eorie des jeux : TD 6    ;      Ann\'ee 2011-12, 2nd Semestre}
%\maketitle
\medskip
\medskip
\medskip

\begin{center}
\textbf{Exercice : La bataille des sexes}\\
\end{center}

Un couple doit d\'ecider comment organiser leur soir\'ee. Ils ont le choix entre aller \`a un match de football ou \`a l'op\'era. 
Pour les deux, le plus important, c'est de passer la soir\'ee ensemble. 
N\'eanmoins, l'\'epouse a une pr\'ef\'erence pour le football et le mari pour l'op\'era. (A bas les pr\'ejug\'es !)

\begin{itemize}
\item Quelles sont les diff\'erentes strat\'egies ? 
\item Comment peut-on \'evaluer les gains de chacun ? 
\item Comment peut-on repr\'esenter la forme normale de ce jeu ? 
\item Ce jeu est-il \`a somme nulle ?
\end{itemize}


\begin{center}
\textbf{Exercice : Le dilemme du prisonnier}
\end{center}

Deux malfaiteurs sont arr\^et\'es pour un m\^eme forfait et interrog\'es s\'eparemment. Si un des deux prisonniers dénonce l'autre, 
il est remis en liberté alors que le second obtient la peine maximale (5 ans). Si les deux se dénoncent entre eux, ils seront condamnés à une peine plus légère (3 ans).
Si les deux refusent de dénoncer, la peine sera minimale (1 an), faute d'éléments au dossier.\\
Quelles sont les strat\'egies de chacun des prisonniers ? Par quel gain peut on mod\'eliser l’issue des interrogatoires ? 
Quelle est alors la forme normale du ``jeu'' ? Quels sont les  \'equilibres  \'eventuels ?
                                                                


\begin{center}
\textbf{Exercice}
\end{center}

    Pour les jeux suivants indiquer les strat\'egies (pures) prudentes de chaque joueur. Si chacun des joueurs
jouent prudemment, regrettent-ils a posteriori leur choix de strat\'egie ?

\[
\begin{array}{l|rr}
    & y_1 & y_2  \\ \hline
x_1 & (3,3) & (-1,2)\\
x_2 & (2,-1)&(0,0)\\
\end{array} \qquad
\begin{array}{l|rr}
    & y_1 & y_2  \\ \hline
x_1 & (3,3) & (-1,2)\\
x_2 & (2,1)&(0,0)\\
\end{array}
\]
Les joueurs gagnent-ils \`a jouer prudemment ?


\newpage


\begin{center}
\textbf{Exercice}
\end{center}

D\'eterminer l'\'equilibre pour ces jeux \`a somme non nulle. On s'aidera des strat\'egies domin\'ees.
\[
\begin{array}{l|rr}
    & y_1 & y_2  \\ \hline
x_1 & (3,0) & (2,1)\\
x_2 & (0,0)&(3,1)\\
x_3 & (1,1)&(1,0)
\end{array} \qquad
\begin{array}{l|rrr}
    & y_1 & y_2 & y_3 \\ \hline
x_1 & (3,1) & (2,2) & (1,3) \\
x_2 & (2,0)&(1,1)&(2,2)\\
x_3 & (1,0)&(0,3)&(0,1)\\
x_4 & (2,1)&(1,0)&(0,3)\\
x_5 & (1,0)&(2,0)&(4,1)\\
\end{array}
\]

\begin{center}
\textbf{Exercice : La fureur de vivre}
\end{center}

                        
Deux jeunes irresponsables roulent en voiture \`a toute allure l’un vers l’autre jusqu'\`a ce que l’un des deux freine
avant l’autre (ou pas). Leur ``gain'' \`a l'issue du jeu peut \^etre quantifi\'e comme suit : s'ils freinent en m\^eme
temps, le gain est 2 pour chacun. Si l'un freine avant l'autre, son gain est 0 alors que l’autre gagne 4. Si aucun
des deux ne freine c’est l’accident, le gain est -2 pour chacun.
Les  \'equilibres du jeu sont ils prudents pour chacun des joueurs ?

\begin{center}
\textbf{Exercice : Probl\`eme de Monty Hall}
\end{center}
Le jeu oppose un pr\'esentateur (joueur Y) \`a un candidat (joueur X). 
Ce joueur est plac\'e devant trois portes ferm\'ees. 
Derri\`ere l'une d'elles se trouve une voiture (ou tout autre prix magnifique) 
et derri\`ere chacune des deux autres se trouve une ch\`evre (ou tout autre prix sans importance). 
Il doit tout d'abord d\'esigner une porte. 
Puis le pr\'esentateur ouvre une porte qui n'est ni celle choisie par le candidat, ni celle cachant la voiture 
(le pr\'esentateur sait quelle est la bonne porte dès le début). 
Le candidat a alors le droit ou bien d'ouvrir la porte qu'il a choisie initialement, ou bien d'ouvrir la troisi\`eme porte.
                        
\begin{itemize}
 \item D\'eterminer la forme extensive de ce jeu.
 \item Donner la forme normale pour ce jeu.
 \item R\'eduire la forme obtenue. 
\end{itemize}

% \begin{center}
% \textbf{Exercice : Des imp\^ots}
% \end{center}
% 
% Une entreprise a deux soci\'et\'es, Transactions ombrag\'ees et Ecervel\'e. Chacune de ces soci\'et\'es paye, respectivement, 
% en moyenne  4,000,000 euros et 12,000,000 euros d’imp\^ots chaque ann\'ee. Pour chaque soci\'et\'e l’entreprise peut d\'ecider de d\'eclarer les vrais comptes, 
% ou de falsifier les comptes de façon que des imp\^ots \`a payer soit z\'ero. L’administration fiscale ne dispose pas de ressources suffisantes pour 
% contr\^oler les comptes de plus d’une soci\'et\'e chaque ann\'ee. S’il examine la soci\'et\'e avec des fausses d\'eclarations, 
% cette soci\'et\'e est oblig\'ee de payer ses imp\^ots plus un redressement s’\'elevant \`a la moiti\'e du montant des imp\^ots à payer.
% 
% a) Mod\'eliser cette situation comme un jeu o\`u la r\'ecompense est la somme d’argent que les imp\^ots reçoivent (c’est-\`a-dire, d\'efinir les strat\'egies Ii pour 
% l’entreprise et les strat\'egies IIj pour le centre des imp\^ots. Puis faire la matrice 4x2).
% b) Montrer que l’entreprise paye en moyenne 14,000,000 euros d’imp\^ots et trouver la strat\'egie optimale des deux joueurs. 
% c) Supposer maintenant que le redressement est deux fois le montant des imp\^ots \`a payer (au lieu de la moiti\'e).
%  Donner la solution de ce jeu.


\end{document}
