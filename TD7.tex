\documentclass[12pt]{article}
%\documentclass[10pt,a4paper]{book}
\usepackage[utf8 ]{inputenc}
\usepackage[francais]{babel}
\usepackage{hyperref}
\usepackage{epsf}
\usepackage{amsmath}
\usepackage{epsfig}
\usepackage{amssymb}
\usepackage{epic}
\usepackage{stmaryrd}
\usepackage{graphicx}
\pagenumbering{arabic}
\usepackage{hyperref}
\usepackage{verbatim}
\usepackage{bm}
\usepackage[T1]{fontenc}
\usepackage{geometry}

\title{Th\'eorie des jeux : TD 1}
\date{}

\begin{document}
\pagestyle{empty}

\textbf{L1 MASS ;   Th\'eorie des jeux : TD 7    ;      Ann\'ee 2011-12, 2nd Semestre}
%\maketitle
\medskip
\medskip
\medskip


\begin{center}
\textbf{Exercice : Des imp\^ots}
\end{center}

Une entreprise a deux soci\'et\'es, Transactions ombrag\'ees et Ecervel\'e. Chacune de ces soci\'et\'es paye, respectivement, 
en moyenne  4,000,000 euros et 12,000,000 euros d’imp\^ots chaque ann\'ee. Pour chaque soci\'et\'e l’entreprise peut d\'ecider de d\'eclarer les vrais comptes, 
ou de falsifier les comptes de façon que des imp\^ots \`a payer soit z\'ero. L’administration fiscale ne dispose pas de ressources suffisantes pour 
contr\^oler les comptes de plus d’une soci\'et\'e chaque ann\'ee. S’il examine la soci\'et\'e avec des fausses d\'eclarations, 
cette soci\'et\'e est oblig\'ee de payer ses imp\^ots plus un redressement s’\'elevant \`a la moiti\'e du montant des imp\^ots à payer.

\medskip

a) Mod\'eliser cette situation comme un jeu o\`u la r\'ecompense est la somme d’argent que les imp\^ots reçoivent.
%  (c’est-\`a-dire, d\'efinir les strat\'egies Ii pour 
% l’entreprise et les strat\'egies IIj pour le centre des imp\^ots. Puis faire la matrice 4x2).\\
\medskip

b) Montrer que l’entreprise paye en moyenne 14,000,000 euros d’imp\^ots et trouver la strat\'egie optimale des deux joueurs. 
\medskip

c) Supposer maintenant que le redressement est deux fois le montant des imp\^ots \`a payer (au lieu de la moiti\'e).
 Donner la solution de ce jeu.


\end{document}
