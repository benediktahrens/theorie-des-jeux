\documentclass[french,11pt,leqno]{article}

\usepackage{amsmath,amssymb,babel,a4wide,theorem}
\usepackage{enumerate}
%\usepackage[latin1]{inputenc}

%\pagestyle{empty}

\textheight 24 true cm
\textwidth 16.5 true cm
\topmargin -1cm
\oddsidemargin -3 truemm
\evensidemargin 3 truemm

\newcommand{\N}{\mathbb{N}}
\newcommand{\Z}{\mathbb{Z}}
\newcommand{\Q}{\mathbb{Q}}
\newcommand{\R}{\mathbb{R}}
\newcommand{\C}{\mathbb{C}}
\newcommand{\K}{\mathbb{K}}
\newcommand{\Rn}{\mathbb{R}^n}

\newcommand{\dsp}{\displaystyle}

\newcounter{exocount}
\setcounter{exocount}{1}
\newcounter{questcount}

\def\exo{\bigskip\noindent{\bf Exercice \theexocount {} -}
  \addtocounter{exocount}{1} \setcounter{questcount}{1}} 

\def\q{\smallskip\noindent{\bf \number\thequestcount -}
  \addtocounter{questcount}{1}}

\def\indic{\smallskip\noindent\textit{Indication.} }

\begin{document}

\begin{center}
{\bf Universit\'e de Nice-Sophia Antipolis}
\end{center}
\bigskip

\noindent{\bf ANN\'EE UNIVERSITAIRE~:} 2011-12

\noindent{\bf ANN\'EE D'\'ETUDE~:} L1 MASS 

\noindent{\bf MATI\`ERE~:} THEORIE DES JEUX 


\vskip1cm

\noindent {\bf NOM:} \hspace{7cm} {\bf PR\'ENOM:} 


\vskip1cm

\begin{center}
{\bf  INTERROGATION DE TD 2}

{\bf 4 avril 2012 }
\end{center}

\medskip

{\bf \noindent Tous les documents et les calculatrices sont interdits.} 

\smallskip

{\bf \noindent Il vous est toujours demand\'e de justifier avec soin vos r\'eponses.} 

\medskip
\exo 
On consid\`ere la forme normale suivante :
\begin{equation*}
\begin{array}{l|rrrrr}
\mathbf{A} & y_1 & y_2  \\ \hline
x_1&2&-2\\
x_2&-1&3
\end{array}%
\end{equation*}
D\'eterminer la valeur mixte du jeu et les strat\'egies mixtes optimales pour chacun des joueurs. 


\medskip

\exo
On consid\`ere la forme normale suivante :
\begin{equation*}
\begin{array}{l|rrrrr}
\mathbf{B} & y_1 & y_2 & y_3 & y_4 & y_5  \\ \hline
x_1&5&-2&3&7&0\\
x_2&0&5&2&2&6\\
x_3&0&3&2&1&5\\

\end{array}% 
\end{equation*}

\begin{enumerate}[a)]%\setlength{\itemsep}{-1ex}
\item Rappeler les d\'efinitions des strat\'egies domin\'ees pour chacun des deux joueurs $X$ et $Y$. 
\item Apr\`es avoir \'elimin\'e les diff\'erentes strat\'egies domin\'ees, on montrera que 
le jeu r\'eduit de $B$ est $B'$ :
 \begin{equation*}
\begin{array}{l|rrr}
\mathbf{B'} & y_1 & y_2 & y_3 \\ \hline
x_1 & 5 & -2& 3\\
x_2 & 0 & 5& 2\\
\end{array}% 
\end{equation*}
\item Le jeu $B'$ poss\`ede--t--il des points selles ? 
\item Soit $(p_1, p_2, p_3)$ la strat\'egie mixte optimale du joueur X. D\'eterminer les probabilit\'es $p_1$, $p_2$, $p_3$ en justifiant soigneusement. 
On donnera et on s'aidera du graphique correspondant construit \textbf{avec soin}. 
\item En d\'eduire la valeur du jeu.
\item Soit $(q_1, q_2, q_3, q_4, q_5)$ la strat\'egie mixte optimale du joueur Y. D\'eterminer les probabilit\'es $q_1$, $q_2$, $q_3$, $q_4$ et $q_5$ en justifiant soigneusement, on pourra 
s'aider de la valeur du jeu. 
\end{enumerate}

\newpage

\exo
Le propri\'etaire d'une maison poss\`edent 3 chiens pour faire fuir les voleurs. 
Cette maison a deux entr\'ees que nous noterons $A$ et $B$. Un clan de deux brigands tentent de p\'en\'etrer dans la maison.\\
Le clan que nous consid\'erons comme le joueur X peut soit d\'ecider de se s\'eparer et donc de mettre un voleur \`a chaque entr\'ee, soit de rester 
group\'e et donc de mettre les deux voleurs \`a la m\^eme entr\'ee.\\ 
Le propri\'etaire que nous appelerons le joueur Y a choisi de mettre deux chiens \`a une entr\'ee et le troisi\`eme \`a l'autre entr\'ee.\\ 
On consid\`ere que X a gagn\'e (+1) s'il existe au moins une entr\'ee o\`u X a strictement plus de voleurs que le joueur Y a de chiens. Dans le cas contraire, X a perdu (-1)
\begin{enumerate}[a)]
\item D\'ecrire de fa\c con claire les trois strat\'egies de X et les deux strat\'egies de Y.
\item D\'eterminer la matrice de paiement M de ce jeu. 
\end{enumerate}


\medskip

\exo 
Soit la matrice qui repr\'esente un jeu \`a deux joueurs \`a somme nulle :
\begin{equation*}
\begin{array}{l|rrr}
\mathbf{C} & y_1 & y_2 & y_3   \\ \hline
x_1 &3&0&2\\
x_2 &-1&-5&0\\
x_3 &2&1&m
\end{array}% 
\end{equation*}
\begin{enumerate}[a)]
\item Pour quelles valeurs de $m$, le jeu $C$ admet-il une valeur en strat\'egie pure ?
\item Pour les autres valeurs de $m$, apr\`es avoir \'eventuellement r\'eduit la matrice, d\'eterminer la strat\'egie mixte optimale du joueur Y. 
\item En d\'eduire la valeur du jeu en fonction de $m$.
\item D\'eterminer la strat\'egie mixte optimale du joueur X. 
\end{enumerate}


\medskip
\medskip




\end{document}