\documentclass[12pt]{article}
%\documentclass[10pt,a4paper]{book}
\usepackage[utf8 ]{inputenc}
\usepackage[francais]{babel}
\usepackage{hyperref}
\usepackage{epsf}
\usepackage{amsmath}
\usepackage{epsfig}
\usepackage{amssymb}
\usepackage{epic}
\usepackage{stmaryrd}
\usepackage{graphicx}
\pagenumbering{arabic}
\usepackage{hyperref}
\usepackage{verbatim}
\usepackage{bm}
\usepackage[T1]{fontenc}
\usepackage{geometry}

\title{Th\'eorie des jeux : TD 1}
\date{}

\begin{document}
\pagestyle{empty}

\textbf{L1 MASS ;   Th\'eorie des jeux : TD 5    ;      Ann\'ee 2011-12, 2nd Semestre}
%\maketitle
\medskip
\medskip
\medskip

\begin{center}
\textbf{Exercice 1}
\end{center}


\begin{equation*}
A=\begin{pmatrix}
0&2&3\\
3&2&0\\
2&0&2
\end{pmatrix};
B=\begin{pmatrix}
0&3&2\\
2&0&3\\
a&1&0
\end{pmatrix}
\end{equation*}

\begin{itemize}
\item Montrer que le jeu $A$ n'admet pas de valeur en strat\'egie pure. 
\item Ce jeu $A$ vadmet une strat\'egie compl\`etement mixte. On note $(p_1, p_2, p_3)$ la strat\'egie mixte de X. Determiner la strat\'egie optimale de X.
\item Donner la valeur du jeu mixte $A$.
\item En s'aidant de la question pr\'ec\'edente, d\'eterminer la strat\'egie optimale de Y pour $A$, not\'ee $(q_1, q_2, q_3)$
\item Montrer que le jeu $B$ n'a jamais de point selle en strat\'egie pure.
\item Si $a\leq 0$, calculer la valeur mixte de $B$ et les strat\'egies mixtes optimales.
\item Pour quelles valeurs de $a$, le jeu $B$ admet-il un point selle en strat\'egie compl\`etement mixtes ? 
Donner alors sa valeur mixte.
\end{itemize}



\begin{center}
\textbf{Exercice 2}
\end{center}

Deux nations sont en guerre. La premi\`ere du colonel Blotto dispose de 4 arm\'ees, la seconde du comte Baloney de 3 arm\'ees.
Il y a deux fronts, un front au nord, le second au sud. Il n'est pas admis de d\'egarnir compl\`etement l'un des fronts, 
une arm\'ee au minimum doit y \^etre pr\'esente. 
Supposons que Blotto envoie $m_1$ arm\'ees au sud et Baloney, $n_1$. 
\begin{itemize}
 \item $m_1=n_1$, la bataille y est ind\'ecise et son gain est nul pour les deux pays
 \item $m_1<n_1$, Blotto obtient le paiement $- m_1$, Baloney de $m_1$
 \item $m_1>n_1$, Blotto obtient le paiement $n_1$, Baloney de $-n_1$
\end{itemize}
De m\^eme pour le front au nord. Toutefois, si Baloney remporte le front nord, il gagne 1 de plus.\\

\begin{itemize}
\item D\'efinir les diff\'erentes strat\'egies des 2 \og joueurs \fg.
\item Ce jeu est-il \`a somme nulle ? Pourquoi ?
\item Poser X, le colonel Blotto et Y, le colonel Baloney. Donner la forme normale de ce jeu.
\item Montrer que ce jeu n'admet pas de point selle.
\item D\'eterminer les strat\'egies mixtes de chacun des joueurs.  
\end{itemize}

\end{document}
