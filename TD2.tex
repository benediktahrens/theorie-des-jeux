
\documentclass[12pt]{article}
%\documentclass[10pt,a4paper]{book}
\usepackage[utf8 ]{inputenc}
\usepackage[francais]{babel}
\usepackage{hyperref}
\usepackage{epsf}
\usepackage{amsmath}
\usepackage{epsfig}
\usepackage{amssymb}
\usepackage{epic}
\usepackage{stmaryrd}
\usepackage{graphicx}
\pagenumbering{arabic}
\usepackage{hyperref}
\usepackage{verbatim}
\usepackage{bm}
\usepackage[T1]{fontenc}
\usepackage{geometry}

\title{Th\'eorie des jeux : TD 1}
\date{}

\begin{document}
\pagestyle{empty}

\textbf{L1 MASS ;   Th\'eorie des jeux : TD 2    ;      Ann\'ee 2011-12, 2nd Semestre}
%\maketitle
\medskip
\medskip
\medskip

\begin{center}
\textbf{Exercice 1}
\end{center}


Etudier les formes normales suivantes, d\'eterminer les points selles et les strat\'egies optimales pour chacun des joueurs :
\begin{equation*}
\begin{pmatrix}
-5&-3&2&1\\
-3&0&3&2\\
-3&0&-3&-1
\end{pmatrix} \text{ ; }
\begin{pmatrix}
4&0&-2\\
-2&-2&-4\\
0&-4&-2
\end{pmatrix}
\end{equation*}

\medskip
\medskip
\begin{center}
\textbf{Exercice 2}
\end{center}
Pierre et Jean inventent un nouveau jeu. Ils poss\`edent chacun 6 cartes en main : un roi rouge et un noir,
une dame rouge et une noire, un valet rouge et un noir. Le roi remporte sur les deux autres figures. La dame remporte sur un valet. 
Si deux cartes sont de m\^emes valeurs, c'est le rouge qui remporte sur le noir. Si deux m\^emes cartes sont jou\'ees, il y a \'egalit\'e. Le gagnant re\c coit 2 euros du perdant.

\begin{itemize}
\item Donner la forme normale de ce jeu et \'etudier ses points selles. Donner aussi les strat\'egies optimales.
\item Pierre et Jean d\'ecident de modifier les r\`egles : Pierre gagne 2 de plus quand il gagne avec des rouges et Jean gagne 1 en cas d'\'egalit\'e pour les cartes. 
Quel joueur a \'et\'e le plus malin ? 
\item Autre changement de r\`egles : maintenant le valet gagne sur le roi et fait gagner $a$ euros au lieu de 2, $a$, un r\'eel tel que $a>2$. Donner la forme normale. 
\end{itemize}
\medskip
\medskip

\begin{center}
\textbf{Exercice 3 : Un mod\`ele \'economique}
\end{center}
On consid\`ere deux marchands en concurrence sur le march\'e d’un produit dont le co\^ut de production unitaire est de 2 euros. 
Lorsque le produit est propos\'e sur le march\'e à 4 euros, il y a 200 clients pr\^ets à l’acheter ; si le prix propos\'e est de 6 euros, 
il ne reste plus que 100 acheteurs qui accepteront de payer ce prix.\\
Chaque marchand choisit ind\'ependamment et dans l’ignorance du choix de son concurrent de fixer son prix de vente à 4 ou 6 euros. 
Si les deux firmes choisissent le m\^eme prix de vente, elles se partagent le march\'e par moiti\'e (leurs produits étant suppos\'es indiscernables) ; 
en revanche si l’une choisit de vendre \`a 4 euros tandis que sa concurrente tente le prix \'elev\'e, elle obtient la totalit\'e du march\'e.\\
Comment pourrait-on repr\'esenter ce mod\`ele sous la forme d'un jeu \`a somme nulle et donner sa forme normale ?
\medskip
\medskip

\begin{center}
\textbf{Exercice 4 : Quelques param\`etres}
\end{center}
\begin{itemize}
\item Soient a, b, c et d trois r\'eels tels que a et d soient tous deux strictements sup\'erieurs \`a b et c. 
Etudier le jeu suivant, peut-il admettre une valeur \`a strat\'egie pure? Si ce n'est pas le cas, comment pourrait-on modifier l'\'enonc\'e pour qu'il puisse l'admettre ? 
Donner un exemple de valeurs, a, b, c et d.
\begin{equation*}
\begin{pmatrix}
a&b\\
c&d
\end{pmatrix}
\end{equation*}
\item Soient a,b et c trois r\'eels avec b>c. Etudier selon les diff\'erents cas pour a :
\begin{equation*}
\begin{pmatrix}
a&a\\
c&a
\end{pmatrix}\text{ ; }
\begin{pmatrix}
a&b\\
c&a
\end{pmatrix}
\end{equation*}

\item Soit la matrice $G=diag(a_1,...a_n)$ avec $a_1,...,a_n>0$. Etudier ce jeu et donner un exemple de jeu qui admet une valeur \`a strat\'egie pure.

\item Etudier les jeux en fonction du param\`etre $a$ : ces jeux admettent-ils une valeur \`a strat\'egie pure ?

\begin{equation*}
\begin{pmatrix}
1&-2&3&1\\
-2&-1&2&3+a\\
-2&1&2&1\\
1&-1&-1&2
\end{pmatrix} \text{ ; }
\begin{pmatrix}
2&1&-1\\
0&a&-2\\
1&1&2
\end{pmatrix} 
\end{equation*}
\end{itemize}
\medskip
\medskip

\begin{center}
\textbf{Exercice 5 : Jeu du Colonel Blotto}
\end{center}

Deux nations sont en guerre. La premi\`ere dispose de 5 arm\'ees, la seconde de 4 arm\'ees. Il y a deux fronts, un front au nord, le second au sud. 
Il n'est pas admis de d\'egarnir compl\`etement l'un des fronts, une arm\'ee au minimum doit y \^etre pr\'esente. On suppose qu'en cas d'\'egalit\'e num\'erique sur un front, 
la bataille y est ind\'ecise et son gain est nul pour les deux pays. Toutefois, en cas de d\'es\'equilibre, il y a victoire de la nation qui a d\'eploy\'e les force les plus importantes.
Les deux fronts ne sont pas d'importances \'egales : une victoire au nord est \'evalu\'ee \`a une valeur de $a$, une victoire au sud \`a une valeur de $b$.


Comment mod\'eliser la forme normale de ce jeu ? On commencera par d\'efinir les diff\'erentes strat\'egies des 2 \og joueurs \fg.

\medskip

\end{document}
