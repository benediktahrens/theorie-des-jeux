\documentclass[11pt]{scrartcl}
%\documentclass[10pt,a4paper]{book}
\usepackage[utf8 ]{inputenc}
\usepackage[francais]{babel}
\usepackage{hyperref}
\usepackage{epsf}
\usepackage{amsmath}
\usepackage{epsfig}
\usepackage{amssymb}
\usepackage{epic}
\usepackage{stmaryrd}
\usepackage{graphicx}
\pagenumbering{arabic}
\usepackage{hyperref}
\usepackage{verbatim}
\usepackage{bm}
\usepackage[T1]{fontenc}
\usepackage{geometry}

\title{Th\'eorie des jeux : TD 1}
\date{}


\DeclareMathOperator{\Maxmin}{Maxmin}
\DeclareMathOperator{\Minmax}{Minmax}
\DeclareMathOperator{\val}{val}


\begin{document}
\pagestyle{empty}

\textbf{L1 MASS ;   Th\'eorie des jeux : TD 3    ;      Ann\'ee 2011-12, 2nd Semestre}
%\maketitle
\medskip
\medskip
\medskip

% 
% \begin{itemize}
%  \item strategies prudentes + rappel
%  \item strategies dominantes + rappel
%  \item points selles
%  \item donner jeu mixte associ\'e a jeu a 2 strategies (matrice 2 x 2)
%  \item controle 14 dec 2006
% \end{itemize}

Soit $(X,Y,u)$ un jeu. On appelle \emph{strat\'egie prudente de $X$} toute strat\'egie de $X$ qui \emph{minimise le risque}, 
c'est--\`a--dire, toute strat\'egie $x\in X$ tel que
 \[ \inf_{y\in Y} u(x,y) = \sup_{x\in X} \inf_{y\in Y} u(x,y)  \enspace . \]

De m\^eme, on appelle \emph{strat\'egie prudente de $Y$} toute strat\'egie de $Y$ qui minimise le risque, c'est--\`a--dire,
 toute strat\'egie $y \in Y$ tel que
\[ \sup_{x\in X} u(x,y) = \inf_{y\in Y} \sup_{x\in X} u(x,y) \enspace . \]


On dit que la strat\'egie $x_1\in X$ \emph{domine} la strat\'egie $x_2 \in X$, si le gain obtenu en jouant $x_1$ est \emph{toujours} --- c'est--\`a--dire
 pour chaque strat\'egie que $Y$ pourrait jouer ---
superieur \`a celui qu'on obtient en jouant $x_2$, i.e.\ si pour toute strat\'egie $y\in Y$ de l'oppos\'e, 
 \[ u(x_1, y) \geq u(x_2,y) \enspace . \]
Dans ce cas, on dit aussi que $x_2$ \emph{est domin\'ee par} $x_1$.

\begin{center}
\textbf{Exercice 1 : Strat\'egies dominantes}
\end{center}

Trouver les strat\'egies dominantes et r\'eduire les formes normales suivantes :
\begin{equation*}
\begin{pmatrix}
-6&-4&1&0\\
-4&-1&-4&-2\\
-4&-1&2&1
\end{pmatrix} \text{ ; }
\begin{pmatrix}
-2&0&-2&1&1\\
3&1&-1&3&2\\
2&3&0&-2&-4\\
5&2&-4&0&-1
\end{pmatrix}\text{ ; }
\begin{pmatrix}
0&1&-2&-3&-1\\
0&1&0&-2&-1\\
1&-1&1&1&-2\\
0&0&-2&0&0\\
-1&-1&-2&-2&-1
\end{pmatrix}
\end{equation*}
Donner les strat\'egies prudentes de chacun des joueurs.


\begin{center}
\textbf{Exercice}
\end{center}

\begin{equation*}
\begin{pmatrix}
2t&2&0\\
-1&2&t
\end{pmatrix}
\end{equation*}

\begin{itemize}
\item D\'eterminer quand on a une/des strat\'egie(s) dominante(s) pour X et Y et la/lesquelles.
\item Dessiner le graphe de la fonction $\inf_{y \in Y} \sup_{x \in X} u_X(x,y)$ et celui de 
la fonction $ \sup_{x \in X} \inf_{y \in Y} u_X(x,y)$ sur le m\^eme dessin.
\item D\'eterminer les strat\'egies prudentes de X et Y en fonction de t.
\item Quand le jeu a-t-il une valeur en strat\'egie pure et quelle est-elle ?
\end{itemize}

\begin{center}
\textbf{Exercice}
\end{center}

D\'eterminer le jeu mixte associ\'e \`a cette forme normale :
\begin{equation*}
\begin{pmatrix}
2&-2\\
-1&2
\end{pmatrix}
\end{equation*}


\end{document}
