\documentclass[11pt]{scrartcl}
%\documentclass[10pt,a4paper]{book}
% \usepackage[latin1]{inputenc}
\usepackage[francais]{babel}
\usepackage{hyperref}
\usepackage{lmodern}
% \usepackage{epsf}
\usepackage{amsmath}
\usepackage{amsthm}
% \usepackage{epsfig}
\usepackage{amssymb}
% \usepackage{epic}
% \usepackage{stmaryrd}
\usepackage{graphicx}
% \pagenumbering{arabic}

\usepackage{verbatim}
% \usepackage{bm}
\usepackage[T1]{fontenc}
% \usepackage{geometry}
\usepackage{multirow}
\usepackage{hyperref}

\title{Th\'eorie des jeux : Cours 1 \\ {\large Jeux \`a somme nulle, Valeur d'un jeu}}
\author{\url{http://math.unice.fr/~ahrens}}
\date{}

\newtheoremstyle{mydefinition}
 {\topsep} % Platz vor dem Theorem
 {\topsep} % Platz nach dem Theorem
 {} % Schrift im Theorem
 {} % Einrueckung (empty = keine Einrueckung, \parindent = Normale Paragrapheneinrueckung)
 {} % Schrift in der Kopfzeile des Theorems
 {} % Punktation nach dem Theoremkopf
 {10pt} %     Space after thm head (\newline = linebreak)
 {{\bfseries\thmnumber{#2 }\thmname{#1}}{\thmnote{ (#3)}}:} % Kopfformatierung

\theoremstyle{mydefinition}
\newtheorem{definition}{D\'efinition}
\newtheorem{exo}[definition]{Exercice}
\newtheorem{example}[definition]{Exemple}
\newtheorem{rem}[definition]{Remarque}

\newtheoremstyle{myplain}
 {\topsep} % Platz vor dem Theorem
 {\topsep} % Platz nach dem Theorem
 {\itshape} % Schrift im Theorem
 {} % Einrueckung (empty = keine Einrueckung, \parindent = Normale Paragrapheneinrueckung)
 {} % Schrift in der Kopfzeile des Theorems
 {} % Punktation nach dem Theoremkopf
 {10pt} %     Space after thm head (\newline = linebreak)
 {{\bfseries\thmnumber{#2 }\thmname{#1}}\thmnote{ (#3)}:} % Kopfformatierung
\theoremstyle{myplain}
\newtheorem{thm}[definition]{Th\'eor\`eme}


\newcommand{\R}{\ensuremath{\mathbb{R}}}

\DeclareMathOperator{\Maxmin}{Maxmin}
\DeclareMathOperator{\Minmax}{Minmax}

\begin{document}

\maketitle


\begin{definition}[Jeu \`a somme nulle]
 Un \emph{jeu \`a somme nulle} est la donn\'ee de
\begin{itemize}
 \item un ensemble $X = \{x_1,\ldots,x_n\}$ de \emph{strat\'egies} --- qu'on appelera les strat\'egies du joueur $X$
 \item un ensemble $Y = \{y_1,\ldots,y_m\}$ de \emph{strat\'egies} --- qu'on appelera les strat\'egies du joueur $Y$
 \item une fonction $X \times Y \to \R$ qui \`a toute paire de strat\'egies $(x,y)\in X\times Y$ associe le paiement $r\in \R$
       que le joueur $Y$ payera a $X$.
\end{itemize}
\end{definition}

Un tel jeu s'appelle \emph{jeu \`a somme nulle} car la perte d'un joueur \'equivaut toujours au gain de l'autre, c'est--\`a--dire
la somme des gain des deux joueurs est nulle.

\begin{definition}[Forme normale d'un jeu \`a somme nulle]\label{def:normal_form}
  Etant donn\'e le jeu $(X,Y,u)$, on repr\'esente les $n$ strat\'egies de $X$ en ligne 
et les $m$ strat\'egies de $Y$ en colonne, et donc obtient un tableau, par exemple:
\[
\begin{array}{l|rrr}
    & y_1 & y_2 & y_3 \\ \hline
x_1 &  -2 & +1 & -1 \\ 
x_2 &   -2 & -4 & +1
\end{array}
\]
qu'on appelle la \emph{forme normale} du jeu $(X,Y,u)$.
\end{definition}

Vu que les coefficients $u(x,y)$ de la matrice correspondent au gain de $X$ --- et donc \`a la perte de $Y$ --- 
le joueur $X$ essaie de maximiser $u(x,y)$, pendant que $Y$ essaie de le minimiser.
Plus pr\'ecis\'ement, pour chaque strat\'egie $x_i$ que le joueur $X$ peut choisir, le joueur $Y$ essaie de choisir la strat\'egie $y_j$ qui
minimise $\{ u(x_i,y_j) \mid 1 \leq j \leq m \}$.

On raisonne en terme de \emph{risque}: chaque joueur r\'eflechit sur le risque de perte pour chacune des strat\'egies \`a sa
disposition.
Pour l'exemple pr\'ec\'edent :
\[
\begin{array}{l|rrr | r|r}
    & y_1 & y_2 & y_3 & \min - & \Maxmin\\ \hline
x_1 &  -2 & +1 & -1 & -2 & \multirow{2}{*}{-2} \\ 
x_2 &   -2 & -4 & +1 & -4  \\ \hline
\max |   & -2 & +1 & +1 \\ \hline
\Minmax &\multicolumn{3}{c}{-2} 
\end{array}
\]
Joueur $Y$ risque de perdre $-2$ (i.e.\ gagner $2$) en choisissant $y_1$, perdre $+1$ en choisissant $y_2$ (si $X$ choisit $x_1$), et 
      $+1$ en choisissant $y_3$.
Joueur $X$ risque de gagner $-2$ (i.e.\ perdre $2$) en choisissant $x_1$ (si $Y$ choisit $y_1$), et perdre $4$ en choisissant $x_2$
 (si $Y$ choisit $y_2$).

% Pour \emph{minimiser} le risque, $X$ va donc choisir la strategie $x_1$.

\begin{definition}
 On appelle 
\[\Maxmin =-2\] 
le max de la derniere colonne, qui est le Max des minima de chacune des lignes.

De m\^eme, on appelle
\[ \Minmax = -2 \]
le min de la derniere ligne, qui est le Min des maxima de chacune des lignes.
\end{definition}

\begin{definition}[Jeu \`a strat\'egie pure]
  Si pour un jeu $(X,Y,u)$ on a $\Minmax = \Maxmin$, on dit que le jeu \emph{admet une valeur en strat\'egie pure}.

  Dans ce cas, on appelle $\Minmax$ \emph{la valeur du jeu}.
\end{definition}

Le jeu de l'\'exemple en haut admet une valeur en strat\'egie pure.

\begin{exo}
 Modifier l'\'exemple de D\'ef.\ \ref{def:normal_form} tel que le jeu n'admette plus une valeur en strat\'egie pure.
\end{exo}

\vspace{1em}

But: savoir repondre \`a
\begin{itemize}
 \item Qu'est-ce qu'un jeu ?
 \item Qu'est-ce que la forme normale d'un jeu, et comment l'interpreter ?
 \item Etant donn\'e un jeu sous forme normale, admet-il une valeur en strat\'egie pure ?
\end{itemize}

\vspace{1em}

% \includegraphics[width=.3\textwidth, angle=270]{calvin_hobbes_01}

\end{document}