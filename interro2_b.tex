\documentclass[french,11pt,leqno]{article}

\usepackage{amsmath,babel,a4wide,theorem}
\usepackage{enumerate}
%\usepackage[latin1]{inputenc}
\usepackage[charter]{mathdesign} 


%\pagestyle{empty}

\textheight 24 true cm
\textwidth 16.5 true cm
\topmargin -1cm
\oddsidemargin -3 truemm
\evensidemargin 3 truemm

\newcommand{\N}{\mathbb{N}}
\newcommand{\Z}{\mathbb{Z}}
\newcommand{\Q}{\mathbb{Q}}
\newcommand{\R}{\mathbb{R}}
\newcommand{\C}{\mathbb{C}}
\newcommand{\K}{\mathbb{K}}
\newcommand{\Rn}{\mathbb{R}^n}

\newcommand{\dsp}{\displaystyle}

\newcounter{exocount}
\setcounter{exocount}{1}
\newcounter{questcount}

\def\exo{\bigskip\noindent{\bf Exercice \theexocount {} -}
  \addtocounter{exocount}{1} \setcounter{questcount}{1}} 

\def\q{\smallskip\noindent{\bf \number\thequestcount -}
  \addtocounter{questcount}{1}}

\def\indic{\smallskip\noindent\textit{Indication.} }

\pagestyle{empty}

\begin{document}

\begin{center}
{\bf Universit\'e de Nice-Sophia Antipolis}
\end{center}
\bigskip

\flushright

\noindent{\bf ANN\'EE UNIVERSITAIRE~:} 2011-12

\noindent{\bf ANN\'EE D'\'ETUDE~:} L1 MASS 

\noindent{\bf MATI\`ERE~:} THEORIE DES JEUX 

\flushleft

\vskip1cm

\noindent {\bf NOM:} \hspace{5cm} {\bf PR\'ENOM:} \hspace{4cm} {\bf NOMBRE FEUILLES:}


\vskip1cm

\begin{center}
{\bf  INTERROGATION DE TD 2}

{\bf 06 avril 2012 }
\end{center}

\medskip

{\bf \noindent Tous les documents et les calculatrices sont interdits.} 

\smallskip

{\bf \noindent Il vous est toujours demand\'e de justifier avec soin vos r\'eponses.} 

\medskip



\exo
On consid\`ere le jeu de forme normale suivante:
\begin{equation*}
\begin{array}{l|rr}
\mathbf{A}& y_1&y_2 \\ \hline
x_1& -3&2\\
x_2& 1&-2
\end{array}
\end{equation*}

\begin{enumerate}[a)]%\setlength{\itemsep}{-1ex}
\item D\'eterminer les strat\'egies prudentes des joueurs $X$ et $Y$.

% \item D\'eterminer les strat\'egies domin\'ees des deux joueurs $X$ et $Y$ de $A$. Donner le jeu r\'eduit de $A$, qu'on appellera $A'$. 
      % A' il restent  x_1, x_3, y_1, y_2
\item Le jeu $A$ poss\`ede--t--il des points selles ? 
           % y en a pas
\item Soit $A_m := (X_m, Y_m, u_m)$ le jeu mixte associ\'e au jeu $A$. Repr\'esenter graphiquement l'ensemble des strat\'egies 
       $X_m$ resp.\ $Y_m$ des joueurs $X$ resp.\ $Y$ dans le jeu mixte.
\item Donner la fonction de paiement $u_m : X_m \times Y_m \to \mathbb{R}$.
\item Calculer les strat\'egies mixtes prudentes $P_X \subset X_m$ de $X$. 
       Tracer les graphes des 2 fonctions consid\'er\'ees dans un m\^eme syst\`eme de coordonn\'ees.
       Donner la valeur du jeu pour ces strat\'egies.
\item Calculer les strat\'egies mixtes prudentes $P_Y \subset Y_m$ de $Y$.
        Tracer les graphes des 2 fonctions consid\'er\'ees dans un m\^eme syst\`eme de coordonn\'ees.
       Donner la valeur du jeu pour ces strat\'egies.
\item Le jeu $A_m$ poss\`ede--t--il des points selles ?
\end{enumerate}



\exo 
On consid\`ere le jeu donn\'e par la forme normale suivante:
\begin{equation*}
% \begin{pmatrix}
% \begin{array}{l|rr}
%     & y_1 & y_2  \\ \hline
% x_1 & t+1&1\\
% x_2 & 0&2t\\
% x_3 & -2&-1
% \end{array}
%
\begin{array}{l|rrr}
 \mathbf{B}   & y_1 & y_2 & y_3  \\ \hline
x_1 & 3& -2 & 1\\
x_2 & -6& 5 & -4\\

\end{array}
%
%
% \end{pmatrix}
\end{equation*}

\begin{enumerate}[a)]%\setlength{\itemsep}{-2ex}
\item D\'emontrer que ce jeu $B$ n'admet pas d'\'equilibre en strat\'egie pure.
\item On consid\`ere ainsi le jeu mixte $B_m$ associ\'e \`a $B$.
Calculer les strat\'egies mixtes prudentes $P_X \subset X_m$ de $X$. 
       Tracer les graphes des 3 fonctions consid\'er\'ees dans un m\^eme syst\`eme de coordonn\'ees.
      En d\'eduire la valeur $v(u_m)$ du jeu mixte. \label{bla}
\item Soit $y^*:=(q_1, q_2, q_3)$ une strat\'egie mixte prudente de $Y$, et $x^* = (p_1,p_2)$ une des strat\'egies
   mixtes prudentes de $X$ trouv\'ees en \ref{bla}). Utiliser que $u_m (x^*, y^*) = v(u_m)$ pour calculer $y^*$.
\end{enumerate}




\end{document}