
\documentclass[12pt]{article}
%\documentclass[10pt,a4paper]{book}
\usepackage[utf8 ]{inputenc}
\usepackage[francais]{babel}
\usepackage{hyperref}
\usepackage{epsf}
\usepackage{amsmath}
\usepackage{epsfig}
\usepackage{amssymb}
\usepackage{epic}
\usepackage{stmaryrd}
\usepackage{graphicx}
\pagenumbering{arabic}
\usepackage{hyperref}
\usepackage{verbatim}
\usepackage{bm}
\usepackage[T1]{fontenc}
\usepackage{geometry}

\title{Th\'eorie des jeux : TD 1}
\date{}

\begin{document}
\pagestyle{empty}

\textbf{L1 MASS ;   Th\'eorie des jeux : TD 1    ;      Ann\'ee 2011-12, 2nd Semestre}
%\maketitle
\medskip

\begin{center}
\textbf{Exercice 1}\\
\end{center}

\medskip
Consid\'erons la forme normale suivante :
\begin{equation*}
\begin{pmatrix}
2&3&-1\\
1&-3&-2\\
-1&-2&4
\end{pmatrix}
\end{equation*}

\begin{itemize}
\item Donner le nombre de strat\'egies diff\'erentes pour le joueur X et le joueur Y.
\item Si le joueur X choisit la strat\'egie 3 et le joueur Y la strat\'egie 2, quel est le gain du joueur X ? Quel est le gain du joueur Y pour cette m\^eme strat\'egie ?
\item Quel strat\'egie permettrait au joueur X de gagner le plus ? A quelle condition concernant le joueur Y? M\^emes questions pour le joueur Y.
\item Indiquer pour chacune des strat\'egies du joueur X, les meilleures r\'eponses du joueur Y et inversement. Celles-ci seront repr\'esent\'ees sous forme de max/min.
\item Le jeu admet-il une valeur en strat\'egie pure? 
\end{itemize}
\medskip
M\^emes questions pour la forme normale :
\begin{equation*}
\begin{pmatrix}
1&-5&3&0&-1&2\\
1&3&4&2&1&1\\
0&2&0&-3&6&-1\\
-4&4&3&0&-1&2
\end{pmatrix}
\end{equation*}
\medskip
\medskip

\begin{center}
\textbf{Exercice 2}\\
\end{center}


Paul et Marie jouent au jeu "caillou, papier, ciseau". Simultan\'ement, chacun repr\'esente un de ses trois \'el\'ements avec leur main : le caillou casse le ciseau, le ciseau coupe le papier et le papier entoure le caillou. 
Deux \'el\'ements identiques aboutissent \`a une \'egalit\'e. Le gagnant re\c coit un euro du perdant.
\begin{itemize}
\item Expliquer pourquoi on peut parler ici de jeu \`a somme nulle. 
\item Repr\'esenter ce jeu sous une forme normale en indiquant les diff\'erentes strat\'egies de chacun. 
\item Donner aussi les maxmin correspondants. 
\item Indiquer si ce jeu admet une valeur en strat\'egie pure.
\item Proposer une version de ce jeu qui ne serait pas \`a somme nulle.
\end{itemize}


\medskip

\medskip

\begin{center}
 \textbf{Exercice 3}\\
\end{center}


\medskip
Soient $X=Y=\{-2,-1,1,2\}$ les strat\'egies possibles des joueurs X et Y et soit $u(x,y)=\dfrac{x}{y}-1$, la fonction de paiement associ\'ee.
\begin{itemize}
\item Donner la forme normale associ\'ee et les maxmin.
\item Ce jeu admet-il une valeur \`a strat\'egie pure ?
\end{itemize}


\end{document}

