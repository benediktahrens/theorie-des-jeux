\documentclass[11pt]{scrartcl}
%\documentclass[10pt,a4paper]{book}
% \usepackage[latin1]{inputenc}
\usepackage[francais]{babel}
\usepackage{hyperref}
\usepackage{lmodern}
% \usepackage{epsf}
\usepackage{amsmath}
\usepackage{amsthm}
% \usepackage{epsfig}
\usepackage{amssymb}
% \usepackage{epic}
% \usepackage{stmaryrd}
\usepackage{graphicx}
% \pagenumbering{arabic}

\usepackage{verbatim}
% \usepackage{bm}
\usepackage[T1]{fontenc}
% \usepackage{geometry}
\usepackage{multirow}
\usepackage{hyperref}

\title{Th\'eorie des jeux : Cours 2\\ {\large Points selles, Strat\'egies optimales}}
\author{\url{http://math.unice.fr/~ahrens}}
\date{}



\newtheoremstyle{mydefinition}
 {\topsep} % Platz vor dem Theorem
 {\topsep} % Platz nach dem Theorem
 {} % Schrift im Theorem
 {} % Einrueckung (empty = keine Einrueckung, \parindent = Normale Paragrapheneinrueckung)
 {} % Schrift in der Kopfzeile des Theorems
 {} % Punktation nach dem Theoremkopf
 {10pt} %     Space after thm head (\newline = linebreak)
 {{\bfseries\thmnumber{#2 }\thmname{#1}}{\thmnote{ (#3)}}:} % Kopfformatierung

\theoremstyle{mydefinition}
\newtheorem{definition}{D\'efinition}
\newtheorem{exo}[definition]{Exercice}
\newtheorem{example}[definition]{Exemple}
\newtheorem{rem}[definition]{Remarque}

\newtheoremstyle{myplain}
 {\topsep} % Platz vor dem Theorem
 {\topsep} % Platz nach dem Theorem
 {\itshape} % Schrift im Theorem
 {} % Einrueckung (empty = keine Einrueckung, \parindent = Normale Paragrapheneinrueckung)
 {} % Schrift in der Kopfzeile des Theorems
 {} % Punktation nach dem Theoremkopf
 {10pt} %     Space after thm head (\newline = linebreak)
 {{\bfseries\thmnumber{#2 }\thmname{#1}}\thmnote{ (#3)}:} % Kopfformatierung
\theoremstyle{myplain}
\newtheorem{thm}[definition]{Th\'eor\`eme}


\newcommand{\R}{\ensuremath{\mathbb{R}}}

\DeclareMathOperator{\Maxmin}{Maxmin}
\DeclareMathOperator{\Minmax}{Minmax}
\DeclareMathOperator{\val}{val}

\begin{document}

 \maketitle
% \noindent
% {{\Large Th\'eorie des jeux : Cours 2} \hfill
% {\large Points selles, Strat\'egies optimales}}\\
% {\url{http://math.unice.fr/~ahrens}}\\

\begin{definition}[Point selle = Equilibre de Nash]
 On appelle  \emph{point selle} ou \emph{equilibre de Nash} de la fonction
 \[ X \times Y \to \R \]
 un couple $(x^*, y^*) \in X\times Y$ tels que pour tout $x\in X$ et $y \in Y$
 \begin{equation}  u(x,y^*) \leq u(x^*,y^*) \leq u(x^*,y) \enspace . \label{eq:point_selle}\end{equation}

 On appelle $S$ l'ensemble des points selles d'un jeu.
\end{definition}

\begin{rem}
 Soit le jeu $(X,Y,u)$  donn\'e par sa forme normale. Alors un coefficient de la matrice
 repr\'esentant du jeu est un point selle si et seulement si ce coefficient est
\emph{en m\^eme temps}
 \begin{itemize}
  \item le maximum de sa colonne \emph{et}
  \item le minimum de sa ligne.
 \end{itemize}

\end{rem}



La signification d'un point selle est:
\begin{itemize}
 \item si joueur $X$ change sa strat\'egie de $x^*$ vers une autre strat\'egie $x\in X$, pendant que 
  $Y$ reste sur $y^*$, alors joueur $X$ gagnera moins ou pareil qu'avec la strat\'egie $x^*$, car
     \[u(x,y^*) \leq u(x^*,y^*) \enspace ; \]
\item si joueur $Y$ change sa strat\'egie de $y^*$ vers une autre strat\'egie $y\in Y$, pendant que 
  $X$ reste sur $x^*$, alors joueur $Y$ gagnera moins ou pareil qu'avec la strat\'egie $y^*$, car
     \[u(x^*,y^*) \leq u(x^*,y) \enspace . \]
 (On rappelle que $Y$ gagne plus quand $u$ est plus petit.)
\end{itemize}

\begin{thm}
 Un jeu ne poss\`ede pas n\'ecessairement de point selle. 

Si le jeu $(X,Y,u)$ poss\`ede au moins un point selle, c'est--\`a--dire si $S \neq \emptyset$, alors tous les points selles 
ont le m\^eme paiement. %Ce paiement est \'egal \`a la valeur du jeu (= $\Minmax

Les points selles sont \'echangeables: pour tout $(x_1, y_1)\in S$ et $(x_2, y_2)\in S$, on a aussi
\[ (x_1, y_2) \in S \quad\text{ et }\quad (x_2, y_1) \in S \enspace . \]

Donc il existe un sous--ensemble $O_X$ des strat\'egies de $X$ et un sous--ensemble $O_Y$ des strat\'egies
 de $Y$, appel\'es \emph{ensemble des strat\'egies optimales} de respectivement, $X$ et $Y$ tels que
 \[ S = O_X \times O_Y \enspace . \]
\end{thm}

\begin{thm}
 Si le jeu $(X,Y,u)$ poss\`ede au moins un point selle, c'est--\`a--dire si $S \neq \emptyset$, alors 
   le jeu admet une valeur en strat\'egie pure. Autrement dit, dans ce cas on a que
     \[ \Minmax = \Maxmin \enspace . \]
  En plus, dans ce cas le paiement des points selles (qui est \'egal pour tous les points selles) est \'egal
  \`a la valeur $\Minmax$ du jeu.

\end{thm}

\begin{rem}
 Le th\'eor\`eme pr\'ecedent dit que, pour trouver les points selles, on peut proc\'eder comme ci:
\begin{enumerate}
 \item D\'eterminer la valeur du jeu $\val(u)$ en strat\'egies pures.
   \begin{itemize}
    \item  Si le jeu n'a pas de valeur en strat\'egie pure, i.e.\ si $\Minmax \neq \Maxmin$, alors il n'y a pas de points selles.
    \item Si le jeu admet une valeur en strat\'egie pure, i.e.\ si $\Minmax = \Maxmin:= \val(u)$, alors les candidats $(x,y)$ possibles pour
            les points selles sont ceux avec $u(x,y) = \val(u)$. Dans ce cas on continue avec l'\'etape suivant.
         
   \end{itemize}

 \item Pour tout point $(x,y)$ avec $u(x,y) = \val(u)$ on examine si la d\'efinition d'un point selle  \eqref{eq:point_selle}
       est v\'erifi\'ee.
\textbf{Attention: Un point $(x,y)$ avec $u(x,y) = \val(u)$ n'est pas forcement un point selle. 
          La v\'erification est necessaire, cf.\ Exo \ref{exo:points_selles}.}
\end{enumerate}

\end{rem}


\begin{thm}\label{thm:val_min_max}
 Supposons que le jeu $(X,Y,u)$ poss\`ede au moins un point selle, c'est--\`a--dire $S \neq \emptyset$. Alors une strat\'egie 
 $\bar{x}\in X$ du joueur $X$ est optimale ($\bar{x} \in O_X$) si et seulement si elle verifie:
   \[  \min_{y\in Y} u(\bar{x},y) = \val(u) \enspace . \]

 Une strat\'egie $x\in X$ n'est pas optimale si et seulement si

   \begin{equation}  \min_{y\in Y} u(\bar{x},y) < \val(u) \enspace . \label{eq:1}\end{equation}
 
 De m\^eme, une strat\'egie 
 $\bar{y}\in Y$ du joueur $Y$ est optimale ($\bar{y} \in O_Y$) si et seulement si elle verifie:
   \[  \max_{x\in X} u(x,\bar{y}) = \val(u) \enspace . \]
Une strat\'egie $y\in Y$ n'est pas optimale si et seulement si
\begin{equation}  \max_{x\in X} u(x,\bar{y}) > \val(u) \enspace . \label{eq:2} \end{equation}
   
\end{thm}

\begin{rem}
 Le th\'eor\`eme \ref{thm:val_min_max} justifie de parler de strat\'egie \emph{optimale}: en jouant une strat\'egie $x\in O_X$, 
 le joueur $X$ obtient un gain qui \'equivaut \emph{au moins} la valeur $\val(u)$ du jeu.

 De m\^eme, en choisissant une strat\'egie optimale $y \in O_Y$ est s\^ur de ne pas perdre plus que la valeur $\val(u)$ du jeu, i.e.\
 il gagnera au moins $-\val(u)$.

 Si $X$ choisit une strat\'egie non optimale, le joueur $Y$ peut trouver une strat\'egie $y$ 
 tel que $u(x,y) < \val(u)$ d'apr\`es la propri\'et\'e \eqref{eq:1}, c--a--d.\ $X$ risque de gagner moins que $\val(u)$ 
  (au cas ou $Y$ anticipe l'erreur de $X$).

 De m\^eme, si $Y$ choisit une strat\'egie non optimale, le joueur $X$ peut trouver une strat\'egie $x$ tel que 
  $u(x,y) > \val(u)$ d'apr\`es la propri\'et\'e \eqref{eq:2}, c--a--d.\ $Y$ risque de perdre plus que $\val(u)$ 
  (au cas ou $X$ anticipe l'erreur de $Y$).

%  Vu qu'on suppose que $X$ et $Y$ sont des joueurs rationnels, ils vont chacun choisir des strat\'egies optimales.

\end{rem}
% 
% \begin{rem}
%   La remarque pr\'ecedente dit que les jeux qui admettent une valeur en strat\'egie pure ne valent pas la peine d'\^etre jou\'es.
% \end{rem}

\begin{rem}
 Si on suppose que $X$ et $Y$ sont des joueurs rationnels, 
  ils vont chacun choisir des strat\'egies optimales et donc le gain 
  correspondra forc\'ement \`a la valeur en strat\'egie pure. 
 Le jeu ne vaudra alors pas vraiment la peine d'\^etre jou\'e. 
Par contre, si vous d\'ecidez de jouer avec votre petit fr\`ere qui n'a pas assist\'e \`a ce cours, 
vous avez quand m\^eme des chances de gagner plus que la valeur en strat\'egie pure...
\end{rem}


\begin{example}
  Soit le jeu $(X,Y,u)$ donn\'e par sa forme normale

  \[
\begin{array}{l|rrr}
    & y_1 & y_2 & y_3 \\ \hline
x_1 &  -2 & 5 & -3 \\ 
x_2 &  -1 & 0 & -1 \\
x_3 &  -3 & 2 & -2
\end{array}
\]

Les strat\'egies optimales du joueur $X$, c'est $O_X = \{x_2\}$.
Les strat\'egies optimales du joueur $Y$, c'est $O_Y = \{y_1, y_3\}$.

\end{example}

\begin{exo}\label{exo:points_selles}
  Soit le jeu $(X,Y,u)$ donn\'e par sa forme normale
  \[
\begin{array}{l|rrrr}
    & y_1 & y_2 & y_3 & y_4\\ \hline
x_1 &  -1 & 1 & 1 & -1 \\ 
x_2 &  -2 & -3 & 2 & 2 \\
x_3 &  -2 & 1 & -2 & 1 \\
x_4 &  -1 & -3 & 4 & 1
\end{array}
\]
Ce jeu admet-il une valeur en strat\'egie pure ? 
Calculer les points selles du jeu. 
Montrer que $(x_1,y_4)$ et $(x_4,y_1)$ ne sont pas des points selles.
Quelles sont les strat\'egies optimales de $X$ et $Y$ ?
\end{exo}



Objectif: savoir repondre aux questions suivantes:

Etant donn\'e le jeu $(X,Y,u)$, 
\begin{itemize}
 \item qu'est-ce que c'est, un point selle ? (def)
 \item trouver les points selle du jeu. (exo)
 \item donner une propri\'et\'e rigolote des points selle d'un jeu. (thm)
 \item qu'est-ce que c'est, une strat\'egie optimale du joueur $X$ ? (def)
 \item trouver les strat\'egies optimales du jeu. (exo)
                                           
\end{itemize}


\end{document}