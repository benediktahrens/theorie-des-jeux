\documentclass[french,11pt,leqno]{article}

\usepackage{amsmath,%amssymb,
         babel,a4wide,theorem, verbatim}
\usepackage{enumerate}
%\usepackage[latin1]{inputenc}
\usepackage[charter]{mathdesign} 

\setlength{\topmargin}{-2cm}
\setlength{\textheight}{23cm}

%\pagestyle{empty}

% \textheight 24 true cm
% \textwidth 16.5 true cm
% \topmargin -1cm
% \oddsidemargin -3 truemm
% \evensidemargin 3 truemm

\newcommand{\N}{\mathbb{N}}
\newcommand{\Z}{\mathbb{Z}}
\newcommand{\Q}{\mathbb{Q}}
\newcommand{\R}{\mathbb{R}}
\newcommand{\C}{\mathbb{C}}
\newcommand{\K}{\mathbb{K}}
\newcommand{\Rn}{\mathbb{R}^n}

\newcommand{\dsp}{\displaystyle}

\newcounter{exocount}
\setcounter{exocount}{1}
\newcounter{questcount}

\def\exo{\bigskip\noindent{\bf Exercice \theexocount {} -}
  \addtocounter{exocount}{1} \setcounter{questcount}{1}} 

\def\q{\smallskip\noindent{\bf \number\thequestcount -}
  \addtocounter{questcount}{1}}

\def\indic{\smallskip\noindent\textit{Indication.} }

\pagestyle{empty}

\begin{document}

% \begin{center}
% {\bf Universit\'e de Nice-Sophia Antipolis}
% \end{center}
% \bigskip

% \flushright
L1 MASS 2011--12 \hfill THEORIE DES JEUX 

CONTROLE FINAL --- 04 mai 2012

% \noindent{\bf ANN\'EE UNIVERSITAIRE~:} 2011-12
% \noindent{\bf ANN\'EE D'\'ETUDE~:}  
% \noindent{\bf MATI\`ERE~:} 
% 
% \flushleft

\vskip1cm

\noindent {\bf NOM:} \hspace{6.4cm} {\bf PR\'ENOM:} \\ 
{\bf MOT SECRET:} \hspace{5cm} {\bf NOMBRE FEUILLES:}




% \begin{center}
% {\bf  CONTROLE FINAL}
% 
% {\bf 04 mai 2012 }
% \end{center}

\medskip

{\bf \noindent Tous les documents et les calculatrices sont interdits.} 

\smallskip

{\bf \noindent Il vous est toujours demand\'e de justifier avec soin vos r\'eponses.} 

\medskip


\exo
On consid\`ere le jeu de forme normale suivante.

\begin{equation*}
\begin{comment}
\begin{array}{l|rrrr}
\mathbf{A} & y_1 & y_2 & y_3 & y_4 \\ \hline
x_1& 3&-1& 4 & 2\\
x_2&-1& 0 &3&4\\
x_3&2&-15&0&3\\
x_4&-1&2&3&7
\end{array}
\qquad
\end{comment}
%
%
\begin{array}{l|rrrr}
\mathbf{A}   & y_1 & y_2 & y_3 & y_4\\ \hline
x_1 &  3 & 9 & -13 & -3 \\
x_2 &  6 & 10 & -6 & -6 \\
x_3 &  6 & -3 & 6 & -3 \\
x_4 &  3 & -3 & -3 & -3 \\  % corrected last entry wrt controle 1
\end{array}
%
% \qquad
% %
% \begin{array}{l|rrr}
% \mathbf{C}& y_1&y_2&y_3 \\ \hline
% x_1& 1&1&p\\
% x_2& -1&1&\sqrt{10}\\
% x_3&\sqrt{8}&1& 1
% \end{array}
\end{equation*}

% \begin{enumerate}[a)]%\setlength{\itemsep}{-1ex}
% \item D\'eterminer les strat\'egies domin\'ees des deux joueurs $X$ et $Y$ de $A$. Donner le jeu r\'eduit de $A$, qu'on appellera $A'$. 
%       % A' il restent  x_1, x_3, y_1, y_2
% \item Le jeu $A'$ poss\`ede--t--il des points selles ? 
%            % y en a pas
% \item Donner les strat\'egies prudentes de $A'$ pour chacun des joueurs.
% \item Comment pourrait--on modifier la fonction de paiement de $A'$ pour obtenir un jeu \`a valeur en strat\'egie pure ? On modifiera
%         une seule valeur de la fonction de paiement.
D\'eterminer les points selles de $\mathbf{A}$. En d\'eduire les strat\'egies optimales de chacun des joueurs.
% \item Selon la valeur de $p$, d\'eterminer les strat\'egies dominantes des deux joueurs $X$ et $Y$ de $C$.
%         % b \leq 1  alors il restent que x_3, y_1    
%         % b \geq 1   pas fait
% \item D\'eterminer, pour quelle valeur de $p$, le jeu $C$ est \`a valeur en strat\'egie pure. Vous donnerez alors les points selles selon les valeurs de $p$.
%         %   minmax = max(b,1)   =?=    min(b,1) =  maxmin 
%         %   pour b = 1 !!!
% \end{enumerate}


\exo
On consid\`ere la forme normale suivante :
\begin{equation*}
\begin{array}{l|rrrrr}
\mathbf{B} & y_1 & y_2 & y_3 & y_4 & y_5  \\ \hline
x_1&5&-2&3&7&0\\
x_2&0&5&2&2&6\\
x_3&0&3&2&1&5\\

\end{array}% 
\end{equation*}

\begin{enumerate}[a)]%\setlength{\itemsep}{-1ex}
\item Rappeler les d\'efinitions des strat\'egies domin\'ees pour chacun des deux joueurs $X$ et $Y$. 
\item Apr\`es avoir \'elimin\'e les diff\'erentes strat\'egies domin\'ees, on montrera que 
le jeu r\'eduit de $\mathbf{B}$ est $\mathbf{B'}$ :
 \begin{equation*}
\begin{array}{l|rrr}
\mathbf{B'} & y_1 & y_2 & y_3 \\ \hline
x_1 & 5 & -2& 3\\
x_2 & 0 & 5& 2\\
\end{array}% 
\end{equation*}
\item Le jeu $\mathbf{B'}$ poss\`ede--t--il des points selles ? 
\item Calculer les strat\'egies \emph{mixtes} prudentes du joueur $X$. 
% 
% \item Soit $(p_1, p_2, p_3)$ la strat\'egie mixte optimale du joueur X. D\'eterminer les probabilit\'es $p_1$, $p_2$, $p_3$ en justifiant soigneusement. 
On donnera et on s'aidera du graphique correspondant construit \textbf{avec soin}. 
\item En d\'eduire la valeur du jeu.
\item Calculer les strat\'egies mixtes optimales du joueur $Y$, en justifiant soigneusement. On pourra 
s'aider de la valeur du jeu. 
\end{enumerate}

\exo
\[
\begin{array}{l|rrr}
\mathbf{C} & y_1 & y_2 & y_3  \\ \hline
x_1& -4& 3& 6 \\
x_2& 8& 3 &  2\\
\end{array}
\]
\begin{enumerate}[a)]
\item Le jeu $\mathbf{C}$ admet--il une valeur en strat\'egie pure ?
\item Calculer les strag\'egies mixtes prudentes de $X$. En d\'eduire la valeur (mixte) du jeu.
\item Calculer les strag\'egies mixtes prudentes de $Y$.
\end{enumerate}


\exo
Le propri\'etaire d'une maison poss\`edent 3 chiens pour faire fuir les voleurs. 
Cette maison a deux entr\'ees que nous noterons $A$ et $B$. Un clan de deux brigands tentent de p\'en\'etrer dans la maison.\\
Le clan que nous consid\'erons comme le joueur X peut soit d\'ecider de se s\'eparer et donc de mettre un voleur \`a chaque entr\'ee, soit de rester 
group\'e et donc de mettre les deux voleurs \`a la m\^eme entr\'ee.\\ 
Le propri\'etaire que nous appelerons le joueur Y a choisi de mettre deux chiens \`a une entr\'ee et le troisi\`eme \`a l'autre entr\'ee.\\ 
On consid\`ere que X re\c coit le gain +1 s'il existe au moins une entr\'ee o\`u X a strictement plus de voleurs que le joueur Y a de chiens. 
Dans le cas contraire, X perd et doit payer 1.
\begin{enumerate}[a)]
\item D\'ecrire de fa\c con claire les trois strat\'egies de X et les deux strat\'egies de Y.
\item D\'eterminer la matrice de paiement M de ce jeu. 
\end{enumerate}


\exo 
On consid\`ere la forme normale $u_t$ suivante, pour $t$ un r\'eel :
\begin{equation*}
% \begin{pmatrix}
 \begin{array}{l|rr}
     & y_1 & y_2  \\ \hline
 x_1 & t+1&1\\
 x_2 & 0&2t\\
 x_3 & -2&-1
 \end{array}
%
% 
% \begin{array}{l|rrr}
%     & y_1 & y_2 & y_3  \\ \hline
% x_1 & t-1&0 & 1\\
% x_2 & 1&3t & 2\\
% 
% \end{array}
%
%
% \end{pmatrix}
\end{equation*}

\begin{enumerate}[a)]%\setlength{\itemsep}{-2ex}
\item Dessiner le graphe de la fonction $t \mapsto \inf_{y \in Y} \sup_{x \in X} u_t(x,y)$. On expliquera les diff\'erentes \'etapes effectu\'ees. \label{question:1}
\item Dessiner sur un autre graphe, la fonction $t \mapsto \sup_{x \in X} \inf_{y \in Y} u_t(x,y)$. \label{question:2}
\item D\'eduire des graphes des questions pr\'ec\'edentes les strat\'egies prudentes de X et Y en fonction de t. 
     On expliquera comment elles sont d\'etermin\'ees.
\item Dessiner sur un troisi\`eme graphe, les fonctions \[t \mapsto \inf_{y \in Y} \sup_{x \in X} u_t(x,y)\] et \[t \mapsto \sup_{x \in X} \inf_{y \in Y} u_t(x,y)\]
 obtenues lors des questions \ref{question:1} et \ref{question:2}. 
\item Pour quelles valeurs de $t$, le jeu a-t-il une valeur en strat\'egie pure et quelle est-elle ? On expliquera comment ces valeurs sont d\'etermin\'ees.
\item Selon les valeurs de $t$, d\'eterminer les strat\'egies dominantes pour $X$. On donnera les arguments utilis\'es.
% \item Selon les valeurs de $t$, d\'eterminer les strat\'egies dominantes pour $Y$. On donnera les arguments utilis\'es.
% TODO


\end{enumerate}





\exo
Deux jeunes irresponsables roulent en voiture \`a toute allure l'un vers l'autre jusqu'\`a ce que l'un des deux freine ou pas. 
Leur ``gain'' \`a l'issue du jeu peut \^etre quantifi\'e comme suit : s'ils freinent en m\^eme
temps, le gain est 2 pour chacun. Si l'un freine avant l'autre, son gain est 0 alors que l'autre gagne 4. Si aucun
des deux ne freine c'est l'accident, le gain est -2 pour chacun.
\begin{enumerate}[a)]
  \item Expliquer pourquoi ce jeu n'est pas \`a somme nulle.
  \item Donner la forme matricielle du jeu.
  \item Donner les \'equilibres du jeu. Expliquer le raisonnement fait pour trouver ces \'equilibres.
  \item Donner les strat\'egies prudentes de chacun des joueurs.
\end{enumerate}


\end{document}