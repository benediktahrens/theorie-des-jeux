\documentclass[12pt]{article}
%\documentclass[10pt,a4paper]{book}
\usepackage[utf8 ]{inputenc}
\usepackage[francais]{babel}
\usepackage{hyperref}
\usepackage{epsf}
\usepackage{amsmath}
\usepackage{epsfig}
\usepackage{amssymb}
\usepackage{epic}
\usepackage{stmaryrd}
\usepackage{graphicx}
\pagenumbering{arabic}
\usepackage{hyperref}
\usepackage{verbatim}
\usepackage{bm}
\usepackage[T1]{fontenc}
\usepackage{geometry}

\title{Th\'eorie des jeux : TD 1}
\date{}

\begin{document}
\pagestyle{empty}

\textbf{L1 MASS ;   Th\'eorie des jeux : TD 4    ;      Ann\'ee 2011-12, 2nd Semestre}
%\maketitle
\medskip
\medskip
\medskip

\begin{center}
\textbf{Exercice 1}
\end{center}


\begin{equation*}
A=\begin{pmatrix}
2&-2\\
-1&2
\end{pmatrix};
B=\begin{pmatrix}
1&2&3&4\\
\frac{3}{2}&0&1&0\\
0&1&4&3
\end{pmatrix}
\end{equation*}
\begin{itemize}
\item Montrer que le jeu $A$ n'admet pas de valeur en strat\'egie pure. D\'eterminer alors les strat\'egies prudentes.
\item D\'eterminer le jeu mixte associ\'e \`a cette forme normale $A$ et sa fonction de paiement.
\item D\'eterminer la valeur mixte de ce jeu $A$ et les strat\'egies optimales associ\'ees pour chacun des joueurs.
\item Montrer que le jeu $B$ n'admet pas de valeur en strat\'egie pure.
\item D\'eterminer la valeur mixte du jeu $B$ et les strat\'egies optimales associ\'ees pour chacun des joueurs. On s'aidera des strat\'egies domin\'ees.
\end{itemize}

\begin{center}
\textbf{Exercice 2}
\end{center}


\begin{equation*}
\begin{pmatrix}
0&2&5&3\\
6&1&0&2
\end{pmatrix}
\end{equation*}
\begin{itemize}
\item Montrer que ce jeu n'admet pas de valeur en strat\'egie pure. D\'eterminer alors les strat\'egies prudentes.
\item D\'eterminer la strat\'egie mixte prudente du joueur X et sa valeur mixte associ\'ee.
\item On s'aidera des r\'esultats de la question pr\'ec\'edente pour d\'eterminer la strat\'egie mixte du joueur Y.
\end{itemize}

\begin{center}
\textbf{Exercice 3}
\end{center}
Soit la matrice qui repr\'esente un jeu \`a deux joueurs \`a somme nulle :
\begin{equation*}
\begin{pmatrix}
4&1&3\\
3&2&m\\
-4&0&1
\end{pmatrix}
\end{equation*}
Discuter l'existence de strat\'egies optimales, pures ou mixtes pour les deux joueurs, en fonction des valeurs du param\`etre $m$.



\end{document}
